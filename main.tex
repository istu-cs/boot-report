\documentclass[11pt]{article}
\usepackage{xltxtra}
\usepackage{polyglossia}
\setdefaultlanguage[spelling=modern]{russian}
%\setmainfont[Mapping=tex-text]{DejaVu Sans}
%\setmainfont[Mapping=tex-text]{Liberation Sans}
\setmonofont[Mapping=tex-text]{DejaVu Sans Mono}
\setmainfont[Mapping=tex-text]{Linux Libertine O}
%\setmonofont[Mapping=tex-text]{Liberation Mono}

\usepackage{verbatim}
\usepackage{tabularx}
\usepackage{float}
\usepackage{url}

\usepackage{indentfirst}

\usepackage{algorithm}
\usepackage{algorithmic}

\usepackage[a4paper]{geometry}
\geometry{left=25mm}
\geometry{right=25mm}
\geometry{top=25mm}
\geometry{bottom=15mm}

\usepackage{listings}

\setlength{\parindent}{10mm}
\makeatletter
% Переопределение команды секции
\renewcommand{\section}{\@startsection{section}{1}%
{\parindent}{3.25ex plus 1ex minus .2ex}%
{1.5ex plus .2ex}{\bfseries\large\uppercase}}

% Переопределение команды подсекции
\renewcommand{\subsection}{\@startsection{subsection}{2}%
{\parindent}{3.25ex plus 1ex minus .2ex}%
{1.5ex plus .2ex}{\bfseries}}
\makeatother

\newcommand{\includepicture}[3]{
\begin{figure}[H]
\begin{center}
\leavevmode
%\large{\textbf{#2}}
\includegraphics[width=#3\textwidth]{#1}
\end{center}
\caption{#2}
\end{figure}
}

\lstset{ %
language=C++,                   % the language of the code
basicstyle=\tiny,               % the size of the fonts that are used for the code
numbers=left,                   % where to put the line-numbers
numberstyle=\footnotesize,      % the size of the fonts that are used for the line-numbers
stepnumber=2,                   % the step between two line-numbers. If it's 1, each line
                                % will be numbered
numbersep=5pt,                  % how far the line-numbers are from the code
%backgroundcolor=\color{white},  % choose the background color. You must add \usepackage{color}
showspaces=false,               % show spaces adding particular underscores
showstringspaces=false,         % underline spaces within strings
showtabs=false,                 % show tabs within strings adding particular underscores
frame=single,                   % adds a frame around the code
tabsize=2,                      % sets default tabsize to 2 spaces
captionpos=b,                   % sets the caption-position to bottom
breaklines=true,                % sets automatic line breaking
breakatwhitespace=false,        % sets if automatic breaks should only happen at whitespace
title=\lstname,                 % show the filename of files included with \lstinputlisting;
                                % also try caption instead of title
escapeinside={\%*}{*)},         % if you want to add a comment within your code
morekeywords={*,...}            % if you want to add more keywords to the set
}

\begin{document}

\tableofcontents
\pagebreak

\section{Постановка задачи}

При поддержке большого парка машин существует необходимость
автоматизации настройки каждой из машин в отдельности,
т.е. задача администратора в идеальном случае должна
сводиться к однократной настройке системы (вместо
повторения одних и тех же действий многократно на каждой машине).

Кроме того, порой возникает необходимость временного изменения
процесса загрузки определённой группы машин. Например, на время
проведения соревнований, для сбора какой-либо информации о машинах...

Одним из способов решения данной задачи является использование сетевой загрузки.

\section{Методы решения задачи}

\subsection{Введение}
Сетевая загрузка -- разновидность инициализации системы, при которой часть системного
кода загружается с удалённой машины (сервера). Загружаться удалённо может как загрузчик,
так и операционная система. Существует множество способов настройки, при которых
может использоваться как только BIOS или EFI, так и загрузчик с жёсткого диска.

Многое зависит от конкретного оборудования и программного обеспечения.
Часть устаревших BIOS'ов не поддерживают
технологию PXE (Preboot eXecution Environment),
в этом случае требуется установка специального загрузчика на жёсткий диск.
Операционные системы семейства Windows также достаточно сложно настроить для загрузки по сети,
поэтому их проще установить на жёсткий диск (при этом загрузчик может быть удалённым).

\subsection{Идентификация и группировка машин}
Машины могут быть идентифицированы при помощи MAC-адреса.
% TODO Стоит ли указать, что это частный случай и можно использовать любой ключ?
Имея таблицу соответствия MAC-адреса и описания самой машины
(например, аудитория + номер) администратор может без труда
объединять нужные машины в группы. В пределах одной группы
машины загружаются одинаково. Таким образом, группе соответствует
конфигурация загрузки.

\vspace{1cm}
\includepicture{groups}{Группы машин}{1}

\section{Реализация}

\subsection{Загрузчики}
%FIXME переехали на iPXE only

Для осуществления загрузки по сети используется загрузчик iPXE.
Он поддерживает гибкую настройку экрана приветствия,
а также позволяет загружать различные операционные системы (в том числе Windows и GNU/Linux).
Также стоит отметить возможность загрузки образов по протоколу HTTP,
что позволяет надёжно передавать данные (в отличие от TFTP).
HTTP также позволяет передать серверу всю необходимую информацию в адресе (в том числе MAC-адрес машины).

\subsection{Процесс загрузки}
В общем случае все машины можно разделить на два класса:
\begin{itemize}
    \item Поддерживают PXE: для загрузки используется BIOS/EFI/etc.

        В этом случае сценарий загрузки следующий:
        \begin{enumerate}
            \item BIOS получает настройки от DHCP-сервера,
                в том числе путь до образа iPXE, доступного на TFTP-сервере.
            \item BIOS загружает с TFTP-сервера undionly.kpxe (файл загрузчика iPXE)
                и передаёт ему управление
        \end{enumerate}

    \item Не поддерживают PXE: для таких машин требуется установка загрузчика
        с поддержкой PXE на жёсткий диск

        В этом случае сценарий загрузки следующий:
        \begin{enumerate}
            \item BIOS загружает extlinux с жёсткого диска и передаёт ему управление
            \item extlinux загружает iPXE c жёсткого диска и передаёт ему управление
            \item iPXE инициализирует сетевой стек,
                получает настройки от DHCP-сервера и загружает указанный в них загрузчик
        \end{enumerate}
\end{itemize}

Дальнейший сценарий в обоих случаях одинаковый:
\begin{enumerate}
    \item iPXE загружает файл конфигурации с HTTP-сервера,
        который формируется его на основе переданных параметров,
        к примеру при помощи MAC-адреса определяется машина,
        группа машин, и соответствующие ей настройки
    \item если это предусмотрено настройками,
        конфигурационный файл содержит меню выбора параметров загрузки,
        которое iPXE отобразит на экране компьютера
    \item iPXE начинает загрузку указанной конфигурации (на основе выбора администратора
        или пользователя), подгружая недостающие файлы с HTTP-сервера
    \item iPXE передаёт управление операционной системе (ядру)
\end{enumerate}

\subsection{Сервер}
Серверная часть проекта работает под управлением операционной
системы GNU/Linux.

\subsubsection{DHCPD}
В качестве DHCP-сервера используется ISC DHCP.
Он настроен таким образом, что при запросе поля
filename, используемого PXE-клиентами как имя
загрузчика, он возвращает имя загрузчика iPXE
на TFTP-сервере.

\subsubsection{TFTPD}
TFTP-сервер хранит undionly.kpxe -- загрузчик iPXE
со встроенным конфигурационным, смысл которого загрузить
конфигурацию iPXE с HTTP-сервера, передав необходимые параметры
(например, MAC-адрес).

\subsubsection{boot -- менеджер загрузки}

\paragraph{boot-сервер} представляет из себя WEB-сервис, задача которого состоит в
\begin{itemize}
    \item предоставлении доступа к конфигурации для удалённых загрузчиков по протоколу HTTP
    \item предоставлении администраторам удобного инструмента для редактирования профилей загрузки
\end{itemize}

Как было сказано выше, существует необходимость разделения машин на группы,
где каждой группе соответствует определённая конфигурация загрузки. Принадлежность
определённой машины к группе задаёт администратор. Сама же машина идентифицируется ключом,
который передаётся как часть URL. Ключом является MAC-адрес сетевой карты, с которой
происходит загрузка.

\paragraph{Реализован} boot-сервер на языке программирования Python
с использованием фреймворка для WEB-приложений Django.

\subsubsection{Удалённые файловые системы}
Для загрузки GNU/Linux клиентов необходим образ
корневой файловой системы. Он доступен клиентам
по протоколу NBD в режиме только чтения.

Помимо этого, иногда возникает необходимость
сохранять на сервере информацию. К примеру,
при создании образов разделов жёстких дисков
для операционной системы Windows и последующем
копировании их на остальные машины сети
удобно использовать имеющийся дистрибутив GNU/Linux,
загружающийся с сервера и не требующий установки.
Для передачи таких файлов используется протокол NFS --
Network File System.

\subsection{Клиент}

\subsubsection{GNU/Linux}
Операционную систему GNU/Linux можно загрузить по-сети
штатными средствами. Существует довольно много различных
конфигураций, но в нашем случае есть ряд требований:
\begin{itemize}
    \item Клиент не должен иметь возможности изменить образ на сервере.
    \item Клиент должен иметь возможность изменять данные локально.
\end{itemize}

Изложенным выше требованиям соответствует конфигурация, при которой
корень файловой системы находится на сервере и доступен по NBD
в режиме только чтение (read-only). Для реализации второго требования
можно воспользоваться файловой системой aufs3, которая позволяет
наложить на недоступный для записи слой другой слой,
уже доступный для записи. Таким образом мы получим файловую систему
с возможностью изменения файлов. Но благодаря технологии Copy-On-Write (cow),
в верхний перезаписываемый уровень файловой системы копируются только те файлы,
которые клиент пытается изменить. Остальные же файлы доступны из нижнего
неперезаписываемого уровня файловой системы. Таким образом достигается
достаточно высокая эффективность работы с такой файловой системы в сравнении
с полным копированием корневой файловой системы или отдельных директорий,
которые должны быть доступны для перезаписи.

Благодаря лёгкости настройки данной операционной системы, с её помощью реализуется
ещё одна задача: выполнение заданий на клиентской машине. Например,
создание и развёртывание образов Windows, сбор информации об аппаратном
обеспечении, запуск сервера проверки решений системы BACS. Это легко сделать
при помощи boot-сервера.

\subsubsection{Windows}
Операционные системы семейства Windows проще установить на жёсткий
диск по двум причинам:
\begin{enumerate}
    \item Сложность настройки загрузки по-сети (проприетарное ПО).
    \item Необходимость установки различных имён машин в Active Directory,
        но при загрузке по сети образ идентичен. Конечно, существует
        теоретическая возможность вмешаться в процесс загрузки и изменить
        часть настроек ОС, как это происходит с GNU/Linux, но это представляет
        собой проблему для проприетарного ПО.
\end{enumerate}

\section{Конфигурация менеджера загрузки}
Каждой группе машин соответствует определённая конфигурация загрузки.
Конфигурация формируется на основе шаблона и списка элементов меню.
Для удобства менеджер загрузки поддерживает глобальные и локальные параметры,
доступные через объекты GLOBAL и LOCAL.

\paragraph{Пример шаблона}
\begin{verbatim}
#!ipxe

:retry
echo [..] Configuring network...
dhcp || goto retry
echo [OK] Configured!

:start
menu Choose operationg system

item {{menu.name}} {{menu.label}}

choose --default win7 --timeout 3000 target || goto cancel
goto ${target}

:failed
echo Boot failed, dropping you to a shell
shell
goto start

:cancel
echo You cancelled the menu, dropping you to a shell
shell
goto start


:{{menu.name}}
{{menu.render}} || goto failed
goto start

\end{verbatim}

\paragraph{Пример элемента меню для Windows 7}
\begin{description}
    \item[name] win7
    \item[label] Windows 7
    \item[render] \verb=sanboot --no-describe --drive 0x80=
\end{description}

\paragraph{Пример элемента меню для GNU/Linux}
\begin{description}
    \item[name] arch64\_nbd
    \item[label] Archlinux amd64 (NBD)
    \item[render]
        \begin{verbatim}
initrd {{GLOBAL.boot}}/{{GLOBAL.initramfs}}
chain {{GLOBAL.boot}}/{{GLOBAL.vmlinuz}} ip=dhcp ro \
    nbd_host={{GLOBAL.nbd_host}} nbd_name={{LOCAL.nbd_name}} \
    raid=noautodetect panic={{GLOBAL.panic}} \
    max_loop={{GLOBAL.max_loop}}
        \end{verbatim}
\end{description}

\section{Тестовые примеры}

% FIXME outdated

\subsection{Загрузка}
\includepicture{ipxe}{undionly.kpxe}{1}
На рисунке показаны системные сообщения, отображаемые на экран во время выполнения сетевой загрузки.
Наглядно виден процесс и порядок выполняемых действий.
\newpage

\includepicture{menu}{Меню загрузки}{1}
Графическое меню, составленное на основе полученного конфигурационного файла.
\newpage

\includepicture{fetch}{Загрузка GNU/Linux с HTTP-сервера}{1}
iPXE загружает ядро и initramfs операционной системы GNU/Linux с HTTP-сервера.
\newpage

\subsection{Администрирование}
\includepicture{adm_config}{Редактирование конфигурации}{1}
Панель администрирования конфигурационного файла.
На странице отображены основные элементы, определяющие свойства конфигурации:
\begin{itemize}
    \item имя
    \item дополнительные параметры
    \item список загружаемых систем
\end{itemize}
\newpage

\includepicture{adm_machine}{Настройка машины}{1}
Панель настройки машины, позволяющая изменять базовую информацию выбранной машины.
\newpage

\includepicture{adm_menuitem}{Настройка пункта меню}{1}
Страница настройки пунктов меню, состоящая из групп команд с указанием их параметров.
Команды выбираются из списка заранее предопределенных строк.
\newpage

\section{Выводы}
В ходе данной работы была создана многокомпонентная система,
позволяющая значительно облегчить процесс администрирования больших групп машин.

Гибкость построенной системы дает возможность применять её для создания различных
профилей загрузки операционных систем семейства Windows и GNU/Linux.

Открытость всех компонентов и программных платформ системы
значительно облегчает модификацию и добавление новых функций.

Недостатками данной системы являются:
\begin{itemize}
    \item сложность развёртывания в сети: требуется интеграция с DHCP-сервером, а также настройка TFTP и HTTP серверов.
    \item возможность применения только в пределах локальной сети.
\end{itemize}

\end{document}
